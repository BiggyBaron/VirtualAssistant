% This is LLNCS.DOC the documentation file of
% the LaTeX2e class from Springer-Verlag
% for Lecture Notes in Computer Science, version 2.4
\documentclass{llncs}
\usepackage{llncsdoc}
%
\begin{document}
    \title{Context Aware Virtual Assistant with
    Case-Based Conflict Resolution in
    Multi-User Smart Home Environment}
    \author{Bauyrzhan Ospan\inst{1} \and Kenzhegali Nurgaliyev\inst{1} \and Mario Jose Quinde Li Say Tan \inst{2} \and Nawaz Khan \inst{2}}
    \institute{Department of Information Technology, Kazakh University of Technology and Business, Kayim Muhamedhanov 37A, Astana 010000, Kazakhstan \\ \email{{bospan,knurgaliev}@gmail.com}
    \and Department of Computer Science, Middlesex University, The Burroughs, London NW4 4BT, United Kingdom \\ \email{{MQ093,N.X.Khan}@live.mdx.ac.uk}}
    \maketitle
    \begin{abstract}
        In this work we will examine and develop a Virtual Assistant which is the intelligence system that is used as a control
        interface of the Smart Home environment in the Farm Side Middlesex University laboratory. Specifically, the system is
        constructed to give multiple users ability to control devices` statuses by voice or text commands through the dialogue
        interface. Primarily, the main purpose of the system is to create user friendly interface with conflict resolution based
        on user cases. As a result, a Case-Based Reasoning algorithm was created with an innovative Ospan Matrix distance
        measurement module. In addition, Feed-Forward Artificial Neural Networks were used to classify user inputs, voice
        recognition and voice generation instruments were used to implement natural dialogue model of interactions with users.
        Moreover, the user friendly graphical web interface was developed. Further, a set of randomised double-blinded
        evaluation tests were established to check procedure of smart home control actions by multiple users.
    \end{abstract}
    \begin{keywords}
        Smart Home, Case-Based Reasoning, Context Awareness, Multiple User systems, Virtual Assistant
    \end{keywords}
    %
    \section{Introduction}
    %
    Managing a comfortable life could be a challenging task for people with special needs because some of them may need additional
    assistance to control electronic and electric devices in the living are [1]. The next group of users that probably seek
    assistance in Activities of Daily Life (ADL) is younger population while adults are away from home [1]. As a result,
    the Smart Home environments are developed to help users in controlling home and devices states with voice graphical
    interfaces [10] that are often called Virtual Assistants [1] or monitor behaviours of people
    with special needs [17] to classify changes in ADL that then alarm or notify relatives, health and care professionals
    and organizations [9]. Additional finding identified other techniques in the state of art.
    For example, in Smart Home systems there are Ontology-based platforms that allows to analyze user`s actions and
    perform context recognition [14] or visual tools for creating personal preferences table of each user based on semantics
    metadata [11]. As a consequence, majority of the researches is aimed to develop systems that helps alone living users.
    \\Despite number of
    researches in the Smart Home solutions that are automated, adaptive, multifunctional and interactive, there is a lack of
    investigations and development of systems that operates in a conflict resolution between multiple users with different
    needs and priorities. Some researchers in order to solve this dilemma developed context-aware automation platforms
    [13, 16] that resolves conflicts between automated devices such as fire alarms and automated door locks.
    On the other hand, these solutions resolves conflicts only between automated devices based on priorities of their rules.
    The next group of
    developers are focused on preference Rule-Based Reasoning (RBR) to resolve conflicts between users [1]. Even through
    these systems introduces efficient set of rules, they are not able to handle with adaptation to the new user cases or
    specific out-of-rule situations. \\To sum up
    state of art, there are number of systems that monitors and adapts to single users, resolves conflicts between devices
    in single user multi-agent environments and resolves conflicts between users based on if-else rules. On the contrary,
    there is a lack of the Virtual Assistants with an adaptive conflict resolution systems. \\
    In this research, we propose a Virtual Assistant (VA) system that controls Smart Home environment based on the user inputs via
    voice and text based dialogue graphical web user interface for multiple users. As a result, we developed number of
    subsystems to fully implement number of requests. First of all, interaction with the VA has to be user-friendly and
    natural as a human to human interaction. Second, the system has to be able to control and monitor state of the Smart
    Home devices. Finally, the VA has to manage to resolve conflicts between multiple users, and adapt to the new user cases.
    \\To cover all objectives, we introduced and developed the VA with next modules. Firstly, we have developed user friendly
    graphical interface based on web and cloud technologies to give to user intuitive way of interacting with the VA. For this
    reason, we used dynamic web framework, cloud voice recognition and generation technologies and animated emotional icons
    to mimicry natural face emotions in the dialogues. In addition to user friendly interface, we have developed and implemented
    adaptive natural language classification algorithms based on Artificial Neural Network (ANN) to make the VA understand
    different variates of the user commands and argumentation sentences. Therefore, users were able to speak to
    VA in the way they speak with other human. Secondly, we used Farm Side Middlesex University Smart Home laboratory and
    Smart Home devices control unit to manage and monitor Smart Home devices. In order that, we have developed and integrated
    software API to implement communication between Smart Home control unit and VA. The last but not the least, we have
    introduced and developed a conflict resolution system that used Case-Based Reasoning (CBR) to make a decision about
    conflict between multiple users. In order to make decision, VA identifies in dialogue form the argumentation of the user
    in performing an action, analyzes via Ospan Matrix distance k-nearest neighbour (kNN) algorithm similarities between
    this case and previous cases and chooses best-fit case. In other words, VA asks user to say why he or she wants to change
    device status if there conflicting instruction from other user and makes decision what to do with device based on users
    arguments and previous cases.
    \\Furthermore, we have established randomized double-blinded evaluation tests. In order to perform randomized double
    blinded experiment, we developed scenario creating sub-module. Firstly, it creates random scenario for tester by choosing
    conflicting device and arguments for tester by using crypto-secure randomise function based on operation system inner
    clocks. Secondly, it provides such instructions to testers that he or she was
    asked to provide command and arguments in their own words. As a result, it was hidden from us and testers scenarios till
    the beginning of the experiment. In addition, testers were asked to fill evaluation form. To conclude, feedback from
    testers are positive and only weak place of the system is accent recognition task.
    \\This paper presents Virtual Assistant in the multiple user Smart Home environment. The architecture of the system is
    described in the Section II. The system algorithms and sequences are described in the Section III. Section IV demonstrates
    our results and validation. Finally, Section V draws the conclusion and predictions for the future work.
    %
    \section{Architecture}
    %
    There will be architecture.
    %
    \subsection{Graphical User Interface}
    %
    Gui is described here.
    \subsection{Dialogue Component}
    %
    All about ANNs and dialogue manager.
    %
    \subsection{Rule-Based Reasoner}
    %
    \subsection{Case-Based Reasoner}
    %
    \subsection{Device Control Unit}
    %
    \section{Sequences}
    %
    There is sequence diagram.
    %
    \subsection{Regular Scenario}
    %
    \subsection{Conflict Scenario}
    %
    \section{Results and Validation}
    %
    \section{Conclusion and Future Work}
    \section{References}
    \label{refer}
    %
    There are three reference systems available; only one, of course,
    should be used for your contribution. With each system (by
    number only, by letter-number or by author-year) a reference list
    containing all citations in the
    text, should be included at the end of your contribution placing the
    \LaTeX{} environment \verb|thebibliography| there.
    For an overall information on that environment
    see the {\em \LaTeX{} User's Guide \& Reference
    Manual\/} by Leslie Lamport, p.~71.

    There is a special {\sc Bib}\TeX{} style for LLNCS that works along
    with the class: \verb|splncs.bst|
    -- call for it with a line \verb|\bibliographystyle{splncs}|.
    If you plan to use another {\sc Bib}\TeX{} style you are customed to,
    please specify the option \verb|[oribibl]| in the
    \verb|documentclass| line, like:
    \begin{verbatim}
        \documentclass[oribibl]{llncs}
    \end{verbatim}
    This will retain the original \LaTeX{} code for the bibliographic
    environment and the \verb|\cite| mechanism that many {\sc Bib}\TeX{}
    applications rely on.
    %
    \subsection{References by Letter-Number or by Number Only}
    %
    References are cited in the text -- using the \verb|\cite|
    command of \LaTeX{} -- by number or by letter-number in square
    brackets, e.g.\ [1] or [E1, S2], [P1], according to your use of the
    \verb|\bibitem| command in the \verb|thebibliography| environment. The
    coding is as follows: if you choose your own label for the sources by
    giving an optional argument to the \verb|\bibitem| command the citations
    in the text are marked with the label you supplied. Otherwise a simple
    numbering is done, which is preferred.
    \begin{verbatim}
        The results in this section are a refined version
        of \cite{clar:eke}; the minimality result of Proposition~14
        was the first of its kind.
    \end{verbatim}
    The above input produces the citation: ``\dots\ refined version of
    [CE1]; the min\-i\-mality\dots''. Then the \verb|\bibitem| entry of
    the \verb|thebibliography| environment should read:
    \begin{verbatim}
        \begin{thebibliography}{[MT1]}
            .
            .
            \bibitem[CE1]{clar:eke}
            Clarke, F., Ekeland, I.:
            Nonlinear oscillations and boundary-value problems for
            Hamiltonian systems.
            Arch. Rat. Mech. Anal. 78, 315--333 (1982)
            .
            .
        \end{thebibliography}
    \end{verbatim}
    The complete bibliography looks like this:
    %
    \begin{thebibliography}{[MT1]}
        %
        \bibitem[CE1]{clar:eke}
        Clarke, F., Ekeland, I.:
        Nonlinear oscillations and
        boundary-value problems for Hamiltonian systems.
        Arch. Rat. Mech. Anal. 78, 315--333 (1982)
        %
        \bibitem[CE2]{clar:eke:2}
        Clarke, F., Ekeland, I.:
        Solutions p\'{e}riodiques, du
        p\'{e}riode donn\'{e}e, des \'{e}quations hamiltoniennes.
        Note CRAS Paris 287, 1013--1015 (1978)
        %
        \bibitem[MT1]{mich:tar}
        Michalek, R., Tarantello, G.:
        Subharmonic solutions with prescribed minimal
        period for nonautonomous Hamiltonian systems.
        J. Diff. Eq. 72, 28--55 (1988)
        %
        \bibitem[Ta1]{tar}
        Tarantello, G.:
        Subharmonic solutions for Hamiltonian
        systems via a $\bbbz_{p}$ pseudoindex theory.
        Annali di Matematica Pura (to appear)
        %
        \bibitem[Ra1]{rab}
        Rabinowitz, P.:
        On subharmonic solutions of a Hamiltonian system.
        Comm. Pure Appl. Math. 33, 609--633 (1980)
    \end{thebibliography}
    %
    \subsubsection*{Number-Only System.}
    %
    For this preferred system do not use the optional argument
    in the \verb|\bibitem| command: then, only numbers will
    appear for the citations in the text (enclosed in square brackets)
    as well as for the marks in your
    bibliography (here the number is only end-punctuated without
    square brackets).

    Subsequent citation numbers in the text are collapsed to ranges.
    Non-numeric and undefined labels are handled correctly but no sorting is
    done.

    E.g., \verb|\cite{n1,n3,n2,n3,n4,n5,foo,n1,n2,n3,?,n4,n5}| -- where
    \verb|n|$x$ is the key of the $x^{\mathrm{th}}$ \verb|\bibitem|
    command in sequence, \verb|foo| is the key of a \verb|\bibitem| with an
    optional argument, and \verb|?| is an undefined reference -- gives
    1,3,2-5,foo,1-3,?,4,5 as the citation reference.

    \begin{verbatim}
        \begin{thebibliography}{1}
            \bibitem {clar:eke}
            Clarke, F., Ekeland, I.:
            Nonlinear oscillations and boundary-value problems for
            Hamiltonian systems.
            Arch. Rat. Mech. Anal. 78, 315--333 (1982)
        \end{thebibliography}
    \end{verbatim}
    %
    \subsection{Author-Year System}
    %
    References are cited in the text by name and year in parentheses
    and should look as follows:
    (Smith 1970, 1980), (Ekeland et al. 1985, Theorem 2), (Jones and Jaffe
    1986; Farrow 1988, Chap.\,2). If the name is part of the sentence
    only the year may appear in parentheses,
    e.g.\ Ekeland et al. (1985, Sect.\,2.1)
    The reference list should contain all citations occurring in the text,
    ordered alphabetically by surname (with initials following). If there
    are several works by the same author(s) the references should be listed
    in the appropriate order indicated below:
    \begin{alpherate}
        \setlength{\hfuzz}{5pt}
        \item
        One author: list works chronologically;
        \item
        Author and same co-author(s): list works chronologically;
        \item
        Author and different co-authors: list works alphabetically
        according to co-authors.
    \end{alpherate}
    If there are several works by the same author(s) and in the same year,
    but which are cited separately, they should be distinguished by the use
    of ``a'', ``b'' etc., e.g.\ (Smith 1982a), (Ekeland et al. 1982b).
    %
    \subsubsection*{How to Code Author-Year System.}
    %
    If you want to use this system you have to specify the option
    \verb|[citeauthoryear]| in the \verb|documentclass|, like:
    \begin{verbatim}
        \documentclass[citeauthoryear]{llncs}
    \end{verbatim}
    Write your citations in the text explicitly except for the year, leaving
    that up to \LaTeX{} with the \verb|\cite| command. Then give only the
    appropriate year as the optional argument (i.e. the label in square
    brackets) with the \verb|\bibitem| command(s).\\[2mm]
    {\itshape Sample Input}
    \begin{verbatim}
        The results in this section are a refined version
        of Clarke and Ekeland (\cite{clar:eke}); the minimality result of
        Proposition~14 was the first of its kind.
    \end{verbatim}
    The above input produces the citation: ``\dots\ refined version of
    Clarke and Ekeland (1982); the minimality\dots''. Then the
    \verb|\bibitem| entry of \verb|clar:eke| in the \verb|thebibliography|
    environment should read:
    \begin{verbatim}
        \begin{thebibliography}{}  % (do not forget {})
            .
            .
            \bibitem[1982]{clar:eke}
            Clarke, F., Ekeland, I.:
            Nonlinear oscillations and boundary-value problems for
            Hamiltonian systems.
            Arch. Rat. Mech. Anal. 78, 315--333 (1982)
            .
            .
        \end{thebibliography}
    \end{verbatim}
    {\itshape Sample Output}
    \bibauthoryear
    %
\end{document}