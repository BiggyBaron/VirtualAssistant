% This is LLNCS.DOC the documentation file of
% the LaTeX2e class from Springer-Verlag
% for Lecture Notes in Computer Science, version 2.4
\documentclass{llncs}
\usepackage{llncsdoc}
%
\begin{document}
    \title{Context Aware Virtual Assistant with
    Case-Based Conflict Resolution in
    Multi-User Smart Home Environment}
    \author{Bauyrzhan Ospan\inst{1} \and Kenzhegali Nurgaliyev\inst{1} \and Mario Jose Quinde Li Say Tan \inst{2} \and Nawaz Khan \inst{2}}
    \institute{Department of Information Technology, Kazakh University of Technology and Business, Kayim Muhamedhanov 37A, Astana 010000, Kazakhstan \\ \email{{bospan,knurgaliev}@gmail.com}
    \and Department of Computer Science, Middlesex University, The Burroughs, London NW4 4BT, United Kingdom \\ \email{{MQ093,N.X.Khan}@live.mdx.ac.uk}}
    \maketitle
    \begin{abstract}
        In this work we will examine and develop a Virtual Assistant which is the intelligence system that is used as a control
        interface of the Smart Home environment in the Farm Side Middlesex University laboratory. Specifically, the system is
        constructed to give multiple users ability to control devices` statuses by voice or text commands through the dialogue
        interface. Primarily, the main purpose of the system is to create user friendly interface with conflict resolution based
        on user cases. As a result, a Case-Based Reasoning algorithm was created with a k-Nearest Neighbor classification algorithm.
        In addition, Feed-Forward Artificial Neural Networks were used to classify user inputs, voice
        recognition and voice generation instruments were used to implement natural dialogue model of interactions with users.
        Moreover, the user friendly graphical web interface was developed. Further, a set of randomised double-blinded
        evaluation tests were established to check procedure of smart home control actions by multiple users.
    \end{abstract}
    \begin{keywords}
        Smart Home, Case-Based Reasoning, Context Awareness, Multiple User systems, Virtual Assistant
    \end{keywords}
    %
    \section{Introduction}
    %
    Managing a comfortable life could be a challenging task for people with special needs because some of them may need additional
    assistance to control electronic and electric devices in the living are [1]. The next group of users that probably seek
    assistance in Activities of Daily Life (ADL) is younger population while adults are away from home [1]. As a result,
    the Smart Home environments are developed to help users in controlling home and devices states with voice graphical
    interfaces [10] that are often called Virtual Assistants [1] or monitor behaviours of people
    with special needs [17] to classify changes in ADL that then alarm or notify relatives, health and care professionals
    and organizations [9]. Additional finding identified other techniques in the state of art.
    For example, in Smart Home systems there are Ontology-based platforms that allows to analyze user`s actions and
    perform context recognition [14] or visual tools for creating personal preferences table of each user based on semantics
    metadata [11]. As a consequence, majority of the researches is aimed to develop systems that helps alone living users.\\
    Despite number of
    researches in the Smart Home solutions that are automated, adaptive, multifunctional and interactive, there is a lack of
    investigations and development of systems that operates in a conflict resolution between multiple users with different
    needs and priorities. Some researchers in order to solve this dilemma developed context-aware automation platforms
    [13, 16] that resolves conflicts between automated devices such as fire alarms and automated door locks.
    On the other hand, these solutions resolves conflicts only between automated devices based on priorities of their rules.
    The next group of
    developers are focused on preference Rule-Based Reasoning (RBR) to resolve conflicts between users [1]. Even through
    these systems introduces efficient set of rules, they are not able to handle with adaptation to the new user cases or
    specific out-of-rule situations. \\
    To sum up
    state of art, there are number of systems that monitors and adapts to single users, resolves conflicts between devices
    in single user multi-agent environments and resolves conflicts between users based on if-else rules. On the contrary,
    there is a lack of the Virtual Assistants with an adaptive conflict resolution systems.
    \section{Methodology}
    In this research, we propose a Virtual Assistant (VA) system that controls Smart Home environment based on the user inputs via
    voice and text based dialogue graphical web user interface for multiple users. As a result, we developed number of
    subsystems to fully implement number of requests. First of all, interaction with the VA has to be user-friendly and
    natural as a human to human interaction. Second, the system has to be able to control and monitor state of the Smart
    Home devices. Finally, the VA has to manage to resolve conflicts between multiple users, and adapt to the new user cases.\\
    To cover all objectives, we introduced and developed the VA with next modules. Firstly, we have developed user friendly
    graphical interface based on web and cloud technologies to give to user intuitive way of interacting with the VA. For this
    reason, we used dynamic web framework, cloud voice recognition and generation technologies and animated emotional icons
    to mimicry natural face emotions in the dialogues. In addition to user friendly interface, we have developed and implemented
    adaptive natural language classification algorithms based on Artificial Neural Network (ANN) to make the VA understand
    different variates of the user commands and argumentation sentences, because they are effective algorithms for
    sentence classification [19]. Therefore, users were able to speak to
    VA in the way they speak with other human. Secondly, we used Farm Side Middlesex University Smart Home laboratory and
    Smart Home devices control unit to manage and monitor Smart Home devices. In order that, we have developed and integrated
    software API to implement communication between Smart Home control unit and VA. The last but not the least, we have
    introduced and developed a conflict resolution system that used Case-Based Reasoning (CBR) to make a decision about
    conflict between multiple users. In order to make decision, VA identifies in dialogue form the argumentation of the user
    in performing an action, analyzes via k-nearest neighbour (kNN) algorithm similarities between
    this case and previous cases and chooses best-fit case [4,7,15]. In other words, VA asks user to say why he or she wants to change
    device status if there conflicting instruction from other user and makes decision what to do with device based on users
    arguments and previous cases. Futhermore, arguments priorities was established by personal user preferences [3] because
    number of researches shows that the preference-based argumentation increases ability of the systems to meet user
    expectations [5].
    \section{Validation}
    Furthermore, we have established randomized double-blinded evaluation tests. In order to perform randomized double
    blinded experiment, we developed scenario creating sub-module. Firstly, it creates random scenario for tester by choosing
    conflicting device and arguments for tester by using crypto-secure randomise function based on operation system inner
    clocks. Secondly, it provides such instructions to testers that he or she was
    asked to provide command and arguments in their own words. As a result, it was hidden from us and testers scenarios till
    the beginning of the experiment. In addition, testers were asked to fill evaluation form. To conclude, feedback from
    testers are positive and only weak place of the system is accent recognition task.\\
    This paper presents Virtual Assistant in the multiple user Smart Home environment. The Virtual Assistant was created
    to manage and monitor Smart Home environment by dialogue based interaction with users. Different machine learning
    algorithms and cloud technologies were implemented to meet project objectives. The main focus of the research was to
    develop adaptive conflict resolution system based on Case-Based Reasoner. The validation tests feedback is positive.
    The future direction of the work is to implement context-aware system that analyzes user behaviour and reports it as
    a case to the database of Case-Based Reasoner.

    \section{References}
    \bibliographystyle{llncs}
    \bibliography{./bib.bib}

\end{document}